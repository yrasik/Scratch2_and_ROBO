\section*{Введение}
\sectionmark{Введение}

Данный проект является очередной реализацией идеи соединить роботизированную платформу с визуальным языком программирования. Подобные проекты: 'ScratchDuino', 'S2A', 'S4A'.

Особенностью является - полностью открытый исходный код всех составных частей проекта, что позволяет добавлять новые функции, совершенствовать проект, не быть привязанным к конкретной аппаратной платформе. Программа - посредник ''Scratch2 to ROBO translator'' отображает возникающие в процессе обмена ошибки, что упрощает отладку программного обеспечения на стороне робота.

Визуальным языком программирования является Scratch. Scratch - стильная, простая и удобная платформа, которая предназначена для простого освоения принципов программирования. Scratch изначально планировался и был разработан для школьников. Scratch позволяет учиться программированию не обладая знаниями английского языка.

Роботизированная платформа в общем случае может быть произвольной. На текущей стадии развития проекта это Arduino. В дальнейшем проект будет развиваться в сторону STM32F + FreeRTOS. 

Arduino — это электронный конструктор и удобная платформа быстрой разработки электронных устройств для новичков и профессионалов. Платформа пользуется огромной популярностью во всем мире благодаря удобству и простоте языка программирования, а также открытой архитектуре и программному коду. Устройство программируется через USB без использования программаторов. Arduino позволяет компьютеру выйти за рамки виртуального мира в физический и взаимодействовать с ним. Устройства на базе  Arduino могут получать информацию об окружающей среде посредством различных датчиков, а также могут управлять различными исполнительными устройствами.

Данный проект может быть полезен в школьных кружках робототехники. Старт не требует значительных капиталловложений. Для старта нужна плата Arduino UNO (китайская), платформа с электродвигателями, платка драйверов электродвигателей, ультразвуковой дальномер, проводки. На www.ebay.com цена за всё это не превышает 25\$.



 