\section{Роборука}
\sectionmark{Роборука}

\subsection{Вариант исполнения роборуки}

Мой вариант исполнения роборуки на 04.10.2016 показан на рисунке~\ref{p:robo_arduino_uno}.

\begin{figure}[H]\center
  \captionsetup{singlelinecheck=true} %центрируем подрисуночную подпись
  \includegraphics*[scale=0.15]{about/images/robohand/robohand_system.jpg}
  \caption{Мой вариант исполнения робота} \label{p:robohand_system}
\end{figure}


\subsection{3-D модели}

3-D модели для печати и STEP-моделив находятся в папке ./3D/RoboHand.

Роборука адаптирована под серврмоторы 'Tower~Pro~9g SG90' и шаговый двигатель '28BYJ-48~–~5V Stepper Motor' (даташиты в папке с 3-D моделями). Как показала практика, данные сервомоторы хлипковаты, но продемонстрировать принцип работы можно.


\subsection{Электроника}

Как видно из рисунка, электроника состоит из двух плат Arduino UNO R3 драйвера шагового двигателя (идёт в комплекте с шаговым двигателем) и блока питания '+5~В  500~мА' (за кадром).

Первая плата Arduino общается с компьютером (с программой 'Scratch2 to ROBO translator') по штатному встроенному интерфейсу (USB-COM). Плата расшифровывает команду и формирует управляющую последовательность по SPI - интерфейсу (SPI-мастер). Скетч лежит в папке './for\_arduino/RoboHand\_program/spi\_master'.

Вторая плата Arduino выступает в роли SPI-slave - исполнителя. Предназначена для приёма команд и преобразованию этих команд в последовательность команд для исполнительных механизмов. Скетч лежит в папке './for\_arduino/RoboHand\_program/spi\_slave\_servo'. 

В пронципе (в других проектах), таких плат-исполнителей может быть несколько, причем навешенных на одни и те же провода (SCK, MOSI, MISO). Проект 'Scratch2 to ROBO translator' не предполагает большое количество однотипных исполнительных механизмов (можно ребёнка окончательно запутать...).























