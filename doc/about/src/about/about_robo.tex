\section{Вариант исполнения роботизированной платформы}
\sectionmark{Вариант исполнения роботизированной платформы}

Мой вариант исполнения робота на 21.06.2016 показан на рисунке~\ref{p:robo_arduino_uno}.

\begin{figure}[H]\center
  \captionsetup{singlelinecheck=true} %центрируем подрисуночную подпись
  \includegraphics*[scale=0.2]{about/images/robo_arduino_uno}
  \caption{Мой вариант исполнения робота} \label{p:robo_arduino_uno}
\end{figure}

Как видно из рисунка, у платы Arduino UNO закончились входы/выходы, электромонтаж оставляет желать лучшего, что пагубно повлияло на связь робота с компьютером через радиоинтерфейс nRF24. Потребляет робот, когда находится без движения - 150~мА по 6,5~В, в движении - 600~мА, так что батарейки только успевай менять...

У робота не хватает клешни на сервоприводах (заявленных в интерфейсе), пьезоизлучателя, светодиодиков...  В общем, буду двигаться по пути: Arduino UNO -> Arduino MEGA~2560 -> STM32F minimal board + FreeRTOS -> Orange~PI.